\documentclass[12pt]{iopart} % Document class declaration

% package "imports"
\usepackage{graphicx}
\usepackage{IEEEtrantools}

% Custom macros
\gdef\mcm{r@{.}l@{ ± }r@{.}l} % Multi Column Measurement; Used for decimal aligning & ± aligning
\gdef\mch#1{\multicolumn{4}{l}{#1}} % Multi Column Header; Used for decimal aligning & ± aligning
\gdef\mcmnd{r@{ ± }l} % Multi Column Measurement No Decimal; Used for ± aligning when the values don't need a decimal point
\gdef\mchnd#1{\multicolumn{2}{l}{#1}} % Multi Column Header No Decimal; Used for  ± aligning when the values don't need a decimal point
\gdef\sci#1#2{#1 \times 10^{#2}}
\gdef\units#1{~\mathrm{#1}}

%%%%%%%%%%%%%%%%%%%% Document Starts %%%%%%%%%%%%%%%%%%%%
\begin{document}

%%%%%%%%%%%%%%%%%%%% Title Page %%%%%%%%%%%%%%%%%%%%
\title{Carnot Cycle Exercise}
\author{Vincent Edwards}
\vspace{10pt}
\begin{indented}
  \item[]Mt.~San Antonio College, Physics 4B, CRN 42240
  \item[]March 27, 2023
  \item[]
  \item[]$T_H = 490~\mathrm{K}$
  \item[]$V_c = \sci{1.90}{-3} \units{m^3}$
\end{indented}
\newpage

%%%%%%%%%%%%%%%%%%%% Purpose %%%%%%%%%%%%%%%%%%%%
\section{Purpose}

The goal of the exercise is to perform various calculations related to the Carnot cycle.

%%%%%%%%%%%%%%%%%%%% Given %%%%%%%%%%%%%%%%%%%%
\section{Given}

\begin{itemize}
  \item $T_H = 490 \units{K}$
  \item $T_C = 300 \units{K}$
  \item $P_c = \sci{1.01}{5} \units{Pa}$
  \item $V_c = \sci{1.90}{-3} \units{m^3}$
  \item $Q_{a \to b} = 300 \units{J}$
  \item $\gamma = 1.40$
  \item $\mathrm{d.o.f.} = 5$
  \item $C_v = \frac 5 2 R$
  \item $C_p = \frac 7 2 R$
\end{itemize}

%%%%%%%%%%%%%%%%%%%% Derivations %%%%%%%%%%%%%%%%%%%%
\section{Derivations}

\subsection{Temperature--Volume Relationship for Adiabatic Process}

\begin{IEEEeqnarray*}{rCl}
  P_i V_i^\gamma & = & P_f V_f^\gamma \\
  P_i V_i V_i^{\gamma-1} & = & P_f V_f V_f^{\gamma-1} \\
  n R T_i V_i^{\gamma-1} & = & n R T_f V_f^{\gamma-1} \\
  T_i V_i^{\gamma-1} & = & T_f V_f^{\gamma-1} 
\end{IEEEeqnarray*}

\subsection{Work by Gas for Isothermal Process}

\begin{IEEEeqnarray*}{rCl}
  W & = & \int\limits_{V=V_i}^{V=V_f} P dV \\
  W & = & \int\limits_{V_i}^{V_f} \frac{nRT}{V} dV \\
  W & = & nRT \ln(V) \vert_{V_i}^{V_f} \\
  W & = & nRT (\ln(V_f) - \ln(V_i)) \\
  W & = & nRT \ln\left( \frac{V_f}{V_i} \right)
\end{IEEEeqnarray*}

%%%%%%%%%%%%%%%%%%%% Results %%%%%%%%%%%%%%%%%%%%
\section{Results}

\begin{table}[htbp]
\caption{\label{tab:pvtkeypoints}
Pressure, Volume, and Temperature for Key Points \\
Note: $T_a = T_b = T_H$ and $T_c = T_d = T_impC$
}
\begin{indented}\lineup\item[]\begin{tabular}{llll}
\br
Point & $P$ (Pa) & $V~\mathrm{(m^3)}$ & $T$ (K) \\
\mr
a & $\sci{1.46}{6}$ & $\sci{2.14}{-4}$ & 490 \\
b & $\sci{5.62}{5}$ & $\sci{5.57}{-4}$ & 490 \\
c & $\sci{1.01}{5}$ & $\sci{1.90}{-3}$ & 300 \\
d & $\sci{2.63}{5}$ & $\sci{7.30}{-4}$ & 300 \\
\br
\end{tabular}\end{indented}\end{table}

\subsection{Moles of Gas ($n$)}

\begin{IEEEeqnarray*}{rCl}
P_c V_c & = & n R T_c \\
n & = & \frac{P_c V_c}{R T_c} \\
n & = & 0.0770 \units{mol} 
\end{IEEEeqnarray*}

\subsection{Pressure ($P_b$) and Volume ($V_b$) at b}

\begin{IEEEeqnarray*}{rCl}
  T_b V_b^{\gamma-1} & = & T_c V_c^{\gamma-1} \\
  V_b & = & V_c \left( \frac{T_c}{T_b} \right)^{\frac{1}{\gamma-1}} \\
  V_b & = & \sci{5.57}{-4} \units{m^3}
\end{IEEEeqnarray*}

\begin{IEEEeqnarray*}{rCl}
  P_b V_b & = & n R T_b \\
  P_b & = & \frac{n R T_b}{V_b} \\
  P_b & = & \sci{5.62}{5} \units{Pa}
\end{IEEEeqnarray*}

\subsection{Pressure ($P_a$) and Volume ($V_a$) at a}

\begin{IEEEeqnarray*}{rCl}
  \Delta U_{a \to b} & = & Q_{a \to b} - W_{a \to b} \\
  0 & = & Q_{a \to b} - nRT_H \ln\left( \frac{V_b}{V_a} \right) \\
  \ln\left( \frac{V_b}{V_a} \right) & = & \frac{Q_{a \to b}}{nRT_H} \\
  \frac{V_b}{V_a} & = & e^{Q_{a \to b}/(nRT_H)} \\
  V_a & = & V_b e^{-Q_{a \to b}/(nRT_H)} \\
  V_a & = & \sci{2.14}{-4} \units{m^3}
\end{IEEEeqnarray*}

\begin{IEEEeqnarray*}{rCl}
  P_a V_a & = & n R T_a \\
  P_a & = & \frac{n R T_a}{V_a} \\
  P_a & = & \sci{1.46}{6} \units{Pa}
\end{IEEEeqnarray*}

\subsection{Pressure ($P_d$) and Volume ($V_d$) at d}

\begin{IEEEeqnarray*}{rCl}
  T_d V_d^{\gamma-1} & = & T_a V_a^{\gamma-1} \\
  V_d & = & V_a \left( \frac{T_a}{T_d} \right)^{\frac{1}{\gamma-1}} \\
  V_d & = & \sci{7.30}{-4} \units{m^3}
\end{IEEEeqnarray*}

\begin{IEEEeqnarray*}{rCl}
  P_d V_d & = & n R T_d \\
  P_d & = & \frac{n R T_d}{V_d} \\
  P_d & = & \sci{2.63}{5} \units{Pa}
\end{IEEEeqnarray*}

%%%%%%%%%%%%%%%%%%%% Conclusion %%%%%%%%%%%%%%%%%%%%
\section{Conclusion}

%%%%%%%%%%%%%%%%%%%% Citations %%%%%%%%%%%%%%%%%%%%
\section{Citations}

\begin{thebibliography}{9}

\bibitem{labpacket}
  Karen Schnurbusch,
  \textit{Physics 4B Lab Book},
  Mt. San Antonio College,
  2023,
  pp. 35-38.

\bibitem{equationsheet}
  Karen Schnurbusch,
  \textit{Physics 4B Equations},
  Mt. San Antonio College,
  2023,
  pp. 1-3.

\end{thebibliography}

\end{document}
%%%%%%%%%%%%%%%%%%%% Document Ends %%%%%%%%%%%%%%%%%%%%
