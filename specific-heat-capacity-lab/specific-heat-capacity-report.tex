\documentclass[12pt]{iopart} % Document class declaration

% package "imports"
\usepackage{graphicx}
\usepackage{IEEEtrantools}

% Custom macros
\gdef\mcm{r@{.}l@{ ± }r@{.}l} % Multi Column Measurement; Used for decimal aligning & ± aligning
\gdef\mch#1{\multicolumn{4}{l}{#1}} % Multi Column Header; Used for decimal aligning & ± aligning

%%%%%%%%%%%%%%%%%%%% Document Starts %%%%%%%%%%%%%%%%%%%%
\begin{document}

\title{Specific Heat Capacity Lab}
\author{Vincent Edwards, Ali Mortada}
\vspace{10pt}
\begin{indented}
  \item[]Mt.~San Antonio College, Physics 4B, CRN 42240
  \item[]March 3, 2023
\end{indented}
\newpage

\section{Purpose}

The goal of the experiment was to determine the specific heat capacities of various substances based on measurements of mass, temperature, power, and/or time.
For parts 1 and 2 the material was known, so the calculated specific heat capacities could be compared to the accepted values.
For part 3 the material was unknown, so its identity was guessed based on the calculated specific heat capacity.

\section{Results}

All three parts of the experiment involved a calculation of the mass of a sample of DI water ($m_w$).
In the experiment, we measured the combined mass of the water and the foam calorimeter it was in ($m_{c+w}$), as well as the mass of the dry calorimeter ($m_c$).
Thus, $m_w$ could be calculated using equation \ref{eq:watermass}.
\begin{IEEEeqnarray}{rCl}
m_w & = & m_{c+w} - m_c \label{eq:watermass}
\end{IEEEeqnarray}
The uncertainty of $m_w$ is given by equation \ref{eq:watermassuncertainty}.
\begin{IEEEeqnarray}{rCl}
\Delta m_w & = & \sqrt{ \left(\frac{\partial m_w}{\partial m_c} \Delta m_c \right)^2 + \left(\frac{\partial m_w}{\partial m_{c+w}} \Delta m_{c+w} \right)^2 }\label{eq:watermassuncertainty} \\
& = & \sqrt{ (\Delta m_c)^2 + (\Delta m_{c+w})^2 } \nonumber
\end{IEEEeqnarray}

In part 1 of the experiment, a sample of DI water stored in a foam calorimeter was heated using an immersion heater for about 2 minutes, and the resulting change in temperature was measured.
Table \ref{tab:part1measurements} contains the measured values.
$m_c$ is the mass of the dry calorimeter measured with an electronic balance.
$m_{c+w}$ is the combined mass of the water and calorimeter measured with an electronic balance.
$T_i$ is the initial temperature of the water measured with a temperature probe.
$H$ is the heat current delivered by the immersion heater measured with a Watt meter.
$t$ is the time the immersion heater was on measured using a stopwatch.
$T_f$ is the final temperature of the water measured using a temperature probe.

Note: explain how uncertainty was approximated.

\begin{table}[htbp]
\caption{\label{tab:part1measurements}
Part 1 Measurements
}
\begin{indented}\lineup\item[]\begin{tabular}{lr@{ ± }l}
\br
Quantity  & \multicolumn{2}{c}{Value} \\
\mr
$m_c$     & 33.02 & 0.01 g \\
$m_{c+w}$ & 439.19 & 0.01 g \\
$T_i$     & 20.2 & 0.4 ℃ \\
$H$       & 293 & 1 W \\
$t$       & 120.2 & 0.2 s \\
$T_f$     & 38.5 & 0.4 ℃ \\
\br
\end{tabular}\end{indented}\end{table}

\begin{table}[htbp]
\caption{\label{tab:uncertaintytrials}
Uncertainty Trials
}
\begin{indented}\lineup\item[]\begin{tabular}{ll}
\br
Run  & $T$ ℃ \\
\mr
1  & 99.0 \\
2  & 99.1 \\
3  & 98.8 \\
4  & 98.7 \\
5  & 98.6 \\
6  & 99.0 \\
7  & 99.1 \\
8  & 99.1 \\
9  & 99.0 \\
10 & 98.9 \\
11 & 98.8 \\
12 & 98.8 \\
13 & 99.1 \\
14 & 99.2 \\
15 & 99.1 \\
16 & 99.2 \\
17 & 99.1 \\
18 & 99.1 \\
19 & 99.2 \\
20 & 99.1 \\
21 & 99.0 \\
22 & 99.1 \\
23 & 99.3 \\
24 & 99.0 \\
25 & 98.8 \\
\br
\end{tabular}\end{indented}\end{table}



\end{document}
%%%%%%%%%%%%%%%%%%%% Document Ends %%%%%%%%%%%%%%%%%%%%
