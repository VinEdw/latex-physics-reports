\documentclass[12pt]{iopart} % Document class declaration

% package "imports"
\usepackage{graphicx}
\usepackage{IEEEtrantools}

% Custom macros
\gdef\mcm{r@{.}l@{ ± }r@{.}l} % Multi Column Measurement; Used for decimal aligning & ± aligning
\gdef\mch#1{\multicolumn{4}{l}{#1}} % Multi Column Header; Used for decimal aligning & ± aligning
\gdef\mcmnd{r@{ ± }l} % Multi Column Measurement No Decimal; Used for ± aligning when the values don't need a decimal point
\gdef\mchnd#1{\multicolumn{2}{l}{#1}} % Multi Column Header No Decimal; Used for  ± aligning when the values don't need a decimal point
\gdef\sci#1#2{#1 \times 10^{#2}}
\gdef\units#1{~\mathrm{#1}}

%%%%%%%%%%%%%%%%%%%% Document Starts %%%%%%%%%%%%%%%%%%%%
\begin{document}

%%%%%%%%%%%%%%%%%%%% Title Page %%%%%%%%%%%%%%%%%%%%
\title{Fiber Optic Project}
\author{Vincent Edwards, Ali Mortada}
\vspace{10pt}
\begin{indented}
  \item[]Mt.~San Antonio College, Physics 4C, CRN 20889
  \item[]Nov 30, 2023
\end{indented}
\newpage

%%%%%%%%%%%%%%%%%%%% Purpose %%%%%%%%%%%%%%%%%%%%
\section{Purpose and Hypothesis}

The goal of the experiment is to examine the relationship between the
speed of light in a vacuum and wavelength of light.
Using the \emph{Industrial Fiber Optics Speed of Light Apparatus} and an
oscilloscope, the time for light to pass through a fiber optic cable of
known length and index of refraction can be measured.
From there, the speed of light can be calculated.
Wavelength will be varied by soldering different colored LEDs onto the apparatus.
In theory, wave speed should be independent of wavelength.

%%%%%%%%%%%%%%%%%%%% Materials %%%%%%%%%%%%%%%%%%%%
\section{Materials}

\begin{itemize}
\item
  Caliper
\item
  Tray

  \begin{itemize}
  \item
    Large enough to support the long fiber optic cable
  \item
    As lightweight as possible
  \end{itemize}
\item
  Balance 1

  \begin{itemize}
  \item
    Able to support the short fiber optic cable
  \item
    More precision, lower weight limit
  \end{itemize}
\item
  Balance 2

  \begin{itemize}
  \item
    Able to support the tray and the long fiber optic cable
    simultaneously
  \item
    Less precision, higher weight limit
  \end{itemize}
\item
  \emph{Industrial Fiber Optics Speed of Light Apparatus}

  \begin{itemize}
  \item
    Also referred to as the ``speed of light kit''
  \item
    Top part is the circuit board, which is screwed onto a base
  \end{itemize}
\item
  110 VAC-to-DC power adapter
\item
  Short fiber optic cable

  \begin{itemize}
  \item
    About 15 cm long
  \end{itemize}
\item
  Long fiber optic cable

  \begin{itemize}
  \item
    About 20 m long
  \end{itemize}
\item
  Dual-channel oscilloscope and its power cable
\item
  2 oscilloscope probes
\item
  Fiber optic red LED

  \begin{itemize}
  \item
    IF-E96E
  \item
    Included with the speed of light kit
  \end{itemize}
\item
  Fiber optic green LED

  \begin{itemize}
  \item
    IF-E93
  \end{itemize}
\item
  Fiber optic blue LED

  \begin{itemize}
  \item
    IF-E92B
  \end{itemize}
\item
  Infrared LED

  \begin{itemize}
  \item
    IR333-A
  \end{itemize}
\item
  Phillips head screwdriver
\item
  Pliers
\item
  Circuit board holder
\item
  Soldering iron
\item
  Solder
\item
  Desoldering solder sucker \textbf{OR} Desoldering wick
\item
  If fiber preparation is required

  \begin{itemize}
  \item
    600-grit polishing paper
  \item
    Polishing liquid (water, glycerin, or light oil)
  \item
    Sharp knife or single-edge razor blade
  \end{itemize}
\end{itemize}

%%%%%%%%%%%%%%%%%%%% Procedures %%%%%%%%%%%%%%%%%%%%
\section{Procedures}

\begin{enumerate}
\def\labelenumi{\arabic{enumi}.}
\item
  Check the ends of the fiber optic cable to see if they are polished
  and flat. If not, they need to be prepared.

  \begin{itemize}
  \item
    Cut 1 to 2 mm off the end of the fiber cable with the sharp knife or
    single-edge razor blade. Try to get as square a cut as possible.
  \item
    Wet the 600-grit polishing paper with the polishing liquid (water,
    glycerin, or light oil) and place the paper, abrasive side facing
    up, on a hard surface.
  \item
    Polish the end of the fiber by moving it in a ``figure 8'' pattern
    while holding it perpendicular against the polishing paper.
  \item
    Check the end of the fiber. If it is cloudy, not flat, or has
    scratches, repeat the previous step.
  \item
    Repeat the previous two steps for both ends of each fiber.
  \end{itemize}
\item
  If needed, desolder the two pins that connect the fiber optic LED to
  the speed of light kit. Afterwards, different fiber optic LEDs can be
  swapped in and out of the slot without re-soldering.

  \begin{itemize}
  \item
    Unscrew the 6 screws connecting the top of the speed of light kit,
    the circuit board, to its base.
  \item
    Remove the screw and nut keeping the LED secure on the circuit
    board. Pliers might be helpful for keeping the nut in place while
    the screw is loosened.
  \item
    Wet with water the sponge used for cooling the tip of the soldering
    iron.
  \item
    Plug in and turn on the soldering iron. Wait for it to get hot
    (about 750°F).
  \item
    Turn on the overhead vent to capture any fumes.
  \item
    Use the circuit board holder to keep the board secure, with the pins
    to be desoldered easily accessible.
  \item
    Use the soldering iron to melt the solder at the pins where the
    fiber optic LED is connected to the circuit board. Use the solder
    sucker or the desoldering wick to remove the solder, and ultimately
    remove the LED. Alternatively, use pliers to gently wiggle and pull
    the LED away from the board. At the same time, use the soldering
    iron to heat each pin.
  \item
    Reattach the LED and secure it with the screw and nut.
  \item
    Reconnect the circuit board to its base, and secure it with the 6
    screws.
  \end{itemize}
\item
  Measure the length of the short fiber optic cable.

  \begin{itemize}
  \item
    Using the caliber, measure the length of the short fiber cable. Make
    sure the cable is straight.
  \end{itemize}
\item
  Measure the mass of both the short and long fiber optic cables.

  \begin{itemize}
  \item
    Using balance 1, measure the mass of the short fiber optic cable.
  \item
    Place the tray on balance 2 and zero the balance.
  \item
    Place the long fiber optic cable on the tray, and measure the mass
    of the cable using balance 2.
  \end{itemize}
\item
  Connect both the speed of light kit and the dual-channel oscilloscope
  to power and to each other.

  \begin{itemize}
  \item
    Connect the oscilloscope to power and turn it on.
  \item
    Connect one of the oscilloscope probes to channel 1, and the other
    probe to channel 2.
  \item
    Connect the probe of channel 1 to the blue test point marked
    ``REFERENCE'' on the speed of light kit. Connect the ground lead of
    the same probe to the test point labeled ``GND'' just below the
    ``REFERENCE'' test point.
  \item
    Connect the probe of channel 2 to the blue test point marked
    ``DELAY'' on the speed of light kit. Connect the ground lead of the
    same probe to the test point labeled ``GND'' just below the
    ``DELAY'' test point.
  \item
    Using the 110 VAC-to-DC power adapter, connect the speed of light
    kit to power.
  \end{itemize}
\item
  Calibrate the speed of light kit using the short fiber optic cable.

  \begin{itemize}
  \item
    Turn the ``Calibration Delay'' knob on the speed of light kit to the
    12 o'clock position.
  \item
    Loosen the fiber optic cinch nuts on the fiber optic LED and
    detector. Insert one end of the short fiber optic cable into the LED
    until it is seated, then lightly tighten the cinch nut. Afterwards,
    insert the other end of the cable into the detector until it is
    seated, then lightly tighten the cinch nut.
  \item
    Set the triggering mode of the oscilloscope to auto, channel 1,
    rising edge. Adjust the trigger level so that the pulses on channels
    1 and 2 can be seen.
  \item
    Adjust the voltage scaling and offset for both channels so that the
    pulses can be clearly seen.
  \item
    Adjust the time scaling and offset such that only one pulse from
    each channel is fully visible on the screen.
  \item
    Using the measure functionality, set the oscilloscope to measure the
    time between the first rising edge of channel 1 and the following
    first rising edge of channel 2 at the 50\% crossing (Channel Delay
    FRFR{[}1-2{]}). Make sure the statistics display is on.
  \item
    Using the measure functionality, set the oscilloscope to measure the
    time between the first rising edge of channel 2 and the following
    first rising edge of channel 1 at the 50\% crossing (Channel Delay
    FRFR{[}2-1{]}). Make sure the statistics display is on.
  \item
    If pulse 1 occurs before pulse 2, then FRFR{[}1-2{]} will give a
    reading and FRFR{[}2-1{]} will not. If pulse 2 occurs before pulse
    1, then FRFR{[}2-1{]} will give a reading and FRFR{[}1-2{]} will
    not. Turning the ``Calibration Delay'' knob on the speed of light
    kit will move pulse 2 left and right, changing the time it occurs.
    Adjust the calibration knob such that the two pulses occur at the
    same time. This will be the case when FRFR{[}1-2{]} and
    FRFR{[}2-1{]} repeatedly toggle between which one is giving a
    reading.
  \end{itemize}
\item
  Measure the time for light to pass through the long fiber optic cable.

  \begin{itemize}
  \item
    Loosen the fiber optic cinch nuts on the fiber optic LED and
    detector. Remove the short fiber optic cable.
  \item
    Insert one end of the long fiber optic cable into the LED until it
    is seated, then lightly tighten the cinch nut. Afterwards, insert
    the other end of the cable into the detector until it is seated,
    then lightly tighten the cinch nut.
  \item
    If needed, adjust the time scaling and offset such that only one
    pulse from each channel is fully visible on the screen.
  \item
    FRFR{[}1-2{]} should be giving a reading, while FRFR{[}2-1{]} should
    not be reading. Turn the statistics menu off and back on, as this
    will clear the data it has measured already. Allow FRFR{[}1-2{]} to
    take at least 1000 time measurements. Then, record the mean and
    standard deviation.
  \item
    With the measurement made, the speed of light kit and oscilloscope
    can be turned off. Loosen the fiber optic cinch nuts and remove the
    long fiber optic cable.
  \end{itemize}
\item
  Switch out the fiber optic LED for one of a different color.

  \begin{itemize}
  \item
    Unscrew the 6 screws connecting the top of the speed of light kit,
    the circuit board, to its base.
  \item
    Remove the screw and nut keeping the LED secure on the circuit
    board. Pliers might be helpful for keeping the nut in place while
    the screw is loosened.
  \item
    Pull out the old fiber optic LED.
  \item
    Attach the new fiber optic LED and secure it with the screw and nut.
  \item
    Reconnect the circuit board to its base, and secure it with the 6
    screws.
  \end{itemize}
\item
  Repeat steps 5-8 for the each fiber optic LED color (red, green,
  blue, infrared).

  \begin{itemize}
  \item
    The kit starts with the red LED installed, so start with that color
    and install it back at the end.
  \end{itemize}
\end{enumerate}

%%%%%%%%%%%%%%%%%%%% Results %%%%%%%%%%%%%%%%%%%%
\section{Results}

Table \ref{tab:short_cable_properties} contains the measured properties of the short fiber optic cable.
$l$ is the length of that cable, and $m$ is the mass of that cable.

\begin{table}[htbp]
\caption{\label{tab:short_cable_properties}
Short Fiber Optic Cable Properties
}
\begin{indented}\lineup\item[]\begin{tabular}{@{}cr@{ ± }l}
\br
  Property & \multicolumn{2}{c}{Measurement} \\
\mr
  $l$      & 148.8 & 0.2 mm \\
  $m$      & 0.57 & 0.01 g  \\
\br
\end{tabular}\end{indented}\end{table}

Table \ref{tab:main_measurements} contains the main measurements made for the various colors of light and long fiber optic cables used.
$\lambda$ is the peak wavelength output by the LED used.
$M$ is the mass of the long fiber optic cable used.
$t$ is the travel time for the light pulse to pass through the long fiber optic cable.

\begin{table}[htbp]
\caption{\label{tab:main_measurements}
Main Measurements \\
Note: UTD means ``unable to detect''
}
\begin{indented}\lineup\item[]\begin{tabular}{@{}ll\mcm\mcm}
\br
  Color    & $\lambda$ (nm) & \mch{$M$ (g)}   & \mch{$t$ (ns)} \\
\mr
  Red      & 645 & 75&10 & 0&01     & 99&85 & 1&53 \\
  Green    & 522 & 75&10 & 0&01     & \multicolumn{4}{l}{UTD through short cable} \\
  Blue     & 470 & 75&10 & 0&01     & \multicolumn{4}{l}{UTD through short cable} \\
  Infrared & 940 & 75&10 & 0&01     & \multicolumn{4}{l}{UTD through long cable} \\
  Infrared & 940 & 37&08 & 0&01     & \multicolumn{4}{l}{UTD through long cable} \\
  Infrared & 940 &  3&75 & 0&01     & \multicolumn{4}{l}{UTD clearly through long cable} \\
  Infrared & 940 &  1&87 & 0&01     & \multicolumn{4}{l}{UTD clearly through long cable} \\
\br
\end{tabular}\end{indented}\end{table}



%%%%%%%%%%%%%%%%%%%% Uncertainty %%%%%%%%%%%%%%%%%%%%
\section{Uncertainty}

%%%%%%%%%%%%%%%%%%%% Conclusion %%%%%%%%%%%%%%%%%%%%
\section{Conclusion}

%%%%%%%%%%%%%%%%%%%% Citations %%%%%%%%%%%%%%%%%%%%
\section{Citations}

\begin{thebibliography}{9}

\bibitem{equationsheet}
  Karen Schnurbusch,
  \textit{Physics 4C Equations},
  Mt. San Antonio College,
  2023,
  pp. 4, 10.

\end{thebibliography}

\end{document}
%%%%%%%%%%%%%%%%%%%% Document Ends %%%%%%%%%%%%%%%%%%%%
