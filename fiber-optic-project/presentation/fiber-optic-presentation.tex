\documentclass{beamer}

% Theme choice
\usetheme{antibes}

% Title page details
\title{Fiber Optic Speed of Light Kit}
\subtitle{Lab Project}
\author{Vincent~Edwards \and Ali~Mortada}
\institute{Mt.~San Antonio College, Physics 4C, CRN 20889}
\date{December 12, 2023}

\begin{document}

% Title slide
\begin{frame}
  \titlepage
\end{frame}

% Outline
\begin{frame}{Outline}
  \tableofcontents
\end{frame}

% Purpose and hypothesis
\section{Purpose and Hypothesis}
\begin{frame}{Purpose and Hypothesis}
  \begin{itemize}
    \item Purpose: Examine the relationship between wavelength ($\lambda$) and speed of light in a vacuum ($c$)
    \item Hypothesis: The speed of light is independent of wavelength
  \end{itemize}
  \begin{center}
    \includegraphics[width=0.75\textwidth]{speed-of-light-apparatus.jpg}
  \end{center}
\end{frame}

% Main Materials
\section{Main Materials}
\begin{frame}{Main Materials}
  \begin{itemize}
    \item Speed of Light Apparatus
    \item Short and long fiber optic cables
    \item Dual-channel oscilloscope
    \item Red, green, blue, and infrared LEDs
    \item Soldering iron and solder sucker
  \end{itemize}
  \begin{center}
    \includegraphics[width=0.8\textwidth]{oscilloscope.jpg}
  \end{center}
\end{frame}

% Procedures
\section{Procedures}
\begin{frame}{Procedures}
  \begin{columns}
    % First column
    \begin{column}{0.6\textwidth}
      \begin{enumerate}
        \item Polish ends of fiber optic cables
        \item Desolder pins connecting LED to the kit
        \item Measure short cable length
        \item Measure mass of both cables
        \item Connect speed of light kit and oscilloscope together
        \item Calibrate kit using the short cable
        \item Measure time for light to travel through long cable
        \item Repeat for different colored LEDs
      \end{enumerate}
    \end{column}

    % Second column
    \begin{column}{0.4\textwidth}
      \includegraphics[width=\textwidth]{fiber-optic-cable.jpg}
    \end{column}
  \end{columns}
\end{frame}

% Equations
\section{Equations}
\begin{frame}{Equations}
  Light slows down in the cable
  \begin{equation}
    c = n v
  \end{equation}

  Long cable length can be determined with linear density
  \begin{equation}
    L = \frac{M}{\mu} = \frac{Ml}{m}
  \end{equation}

  Speed is distance traveled divided by time
  \begin{equation}
    v = \frac{L}{t}    
  \end{equation}

  Combining these equations together
  \begin{equation}
    c = \frac{nMl}{tm}
  \end{equation}
\end{frame}

% Results
\section{Results}
\begin{frame}{Results}
  \begin{itemize}
    \item Red light worked fine ($c = (2.93 \pm 0.07) \times 10^8~\mathrm{m/s}$)
    \item Blue and green light not detected through short cable (photodetector less sensitive to lower visible wavelengths)
    \item Infrared light not detected through long cable (dissipating earlier)
    \item Infrared light not detected clearly through shortened cables (switching time not fast enough)
  \end{itemize}
  \begin{center}
    \includegraphics[height=0.4\textheight]{../detector-response-vs-wavelength.png}
  \end{center}
\end{frame}

% Conclusions
\section{Conclusions}
\begin{frame}{Conclusions}
  \begin{itemize}
    \item Not enough data to support or reject hypothesis
    \item Speed of Light Kit is a one-trick pony
  \end{itemize}
  \begin{center}
    \includegraphics[height=0.6\textheight]{pensive-emoji.png}
  \end{center}
\end{frame}

\end{document}
