\documentclass[12pt]{iopart} % Document class declaration

% package "imports"
\usepackage{graphicx}

%%%%%%%%%%%%%%%%%%%% Document Starts %%%%%%%%%%%%%%%%%%%%
\begin{document}

% Declare the title
\title{Test Report}

% Declare the authors
\author{Vincent Edwards, Ryan Nguyen and Joanne Zhou}

\vspace{10pt}
\begin{indented}
  \item[]Mt.~San Antonio College, Physics 4A, CRN 26537
  \item[]October 7, 2022
  \vspace{10pt}
  \item[]\textbf{Objective}\\
    State what you want to achieve in this experiment.
    This should be the scientific goal of the experiment, not the educational goal (though you should understand that too).
    One or two well thought out sentences is all that you should need for this.
\end{indented}

\section{Theory}
Introduce the relevant equations in this section.
\begin{equation}
  \vec F = m \vec a \label{eq:n2l}
\end{equation}
Throughout the report assume the audience is a reader at your level, perhaps a student taking the same course but at a different school.
Do not fill pages with calculations.
Summarize the key moments in the solving process, enough that another person can fill in the blanks and understand how you got your result.

\section{Method}

Briefly describe what you did in the experiment.
You should demonstrate your understanding of what you measured, how you measured it, and why this measurement is relevant to the objective.
You should also mention and explain anything in particular you did to reduce error.
A diagram of the lab setup should be hand drawn, either digitally or physically.
An example is shown in figure \ref{fig:exampleSetupDiagram}.

\begin{figure}
  \centering
  \includegraphics[width=0.8\textwidth]{example-setup-diagram.png}
  \caption{\label{fig:exampleSetupDiagram}Example Setup Diagram}
\end{figure}

\section{Results}

\Table{\label{tabone}A simple example produced using the standard table commands 
and $\backslash${\tt lineup} to assist in aligning columns on the 
decimal point. The width of the 
table and rules is set automatically by the 
preamble.} 
\br                              
$\0\0A$&$B$&$C$&\m$D$&\m$E$&$F$&$\0G$\cr 
\mr
\0\023.5&60  &0.53&$-20.2$&$-0.22$ &\01.7&\014.5\cr
\0\039.7&\-60&0.74&$-51.9$&$-0.208$&47.2 &146\cr 
\0123.7 &\00 &0.75&$-57.2$&\m---   &---  &---\cr 
3241.56 &60  &0.60&$-48.1$&$-0.29$ &41   &\015\cr 
\br
\endTable

The raw data, calculated data, and graphs should go in here.
Tables should be used to organize data.
Table \ref{tabone} is one such table.
The relationship shown in most graphs should be noted (e.g.~linear, proportional, quadratic, exponential, etc.).
You might even make reference to the equation (\ref{eq:n2l}) the graph is consistent with.
All measurements should have an uncertainty, along with a short note on how that uncertainty estimate was arrived at.

\section{Conclusion}
Summarize the experiments and results.
Discuss whether the results agree with the predictions (the theoretical calculations), making sure to justify using the data.
If not, discuss potential sources of error, as well as improvements that could be made to the experiment that would address those reasons.

\end{document}
%%%%%%%%%%%%%%%%%%%% Document Ends %%%%%%%%%%%%%%%%%%%%
