\documentclass[12pt]{iopart} % Document class declaration

% package "imports"
\usepackage{graphicx}
\usepackage{IEEEtrantools}

% Custom macros
\gdef\mcm{r@{.}l@{ ± }r@{.}l} % Multi Column Measurement; Used for decimal aligning & ± aligning
\gdef\mch#1{\multicolumn{4}{l}{#1}} % Multi Column Header; Used for decimal aligning & ± aligning
\gdef\mcmnd{r@{ ± }l} % Multi Column Measurement No Decimal; Used for ± aligning when the values don't need a decimal point
\gdef\mchnd#1{\multicolumn{2}{l}{#1}} % Multi Column Header No Decimal; Used for  ± aligning when the values don't need a decimal point
\gdef\sci#1#2{#1 \times 10^{#2}}
\gdef\units#1{~\mathrm{#1}}

%%%%%%%%%%%%%%%%%%%% Document Starts %%%%%%%%%%%%%%%%%%%%
\begin{document}

%%%%%%%%%%%%%%%%%%%% Title Page %%%%%%%%%%%%%%%%%%%%
\title{Resistance Lab}
\author{Vincent Edwards, Ali Mortada, Andrew Bringas}
\vspace{10pt}
\begin{indented}
  \item[]Mt.~San Antonio College, Physics 4B, CRN 42240
  \item[]April 26, 2023
\end{indented}
\newpage

%%%%%%%%%%%%%%%%%%%% Purpose %%%%%%%%%%%%%%%%%%%%
\section{Purpose}

For part 1, the goal was to examine how material, cross-sectional area, and length impact resistance.
For each coil of wire, resistance was determined in three ways: based on measurements of voltage and current; based on the resistance reading from the multimeter; and based on the dimensions of the wires and the resistivity of the material.
In addition, the results of the three methods of determining resistance were compared.

%%%%%%%%%%%%%%%%%%%% Results %%%%%%%%%%%%%%%%%%%%
\section{Results}

Table \ref{tab:resistance_coils} contains the properties of the resistance coils used in part 1.
$\rho$ is the resistivity of the material.
$L$ is the length of the coil of wire.
$D$ is the diameter of the wire.

\begin{table}[htbp]
\caption{\label{tab:resistance_coils}
Resistance Coils
}
\begin{indented}\lineup\item[]\begin{tabular}{llrrr}
\br
  Coil & Material      & $\rho~(\mathrm{\Omega~m})$ & $L$ (cm) & $D$ (cm) \\
\mr
  1    & Nickel-Silver & $\sci{44}{-8}$            &       40 & 0.0254 \\
  2    & Nickel-Silver & $\sci{44}{-8}$            &       80 & 0.0254 \\
  3    & Nickel-Silver & $\sci{44}{-8}$            &      120 & 0.0254 \\
  4    & Nickel-Silver & $\sci{44}{-8}$            &      160 & 0.0254 \\
  5    & Nickel-Silver & $\sci{44}{-8}$            &      200 & 0.0254 \\
  6    & Nickel-Silver & $\sci{44}{-8}$            &      200 & 0.0320 \\
  7    & Copper        & $\sci{1.72}{-8}$          &     2000 & 0.0254 \\
\br
\end{tabular}\end{indented}\end{table}

%%%%%%%%%%%%%%%%%%%% Uncertainty %%%%%%%%%%%%%%%%%%%%
\section{Uncertainty}

%%%%%%%%%%%%%%%%%%%% Conclusion %%%%%%%%%%%%%%%%%%%%
\section{Conclusion}

%%%%%%%%%%%%%%%%%%%% Citations %%%%%%%%%%%%%%%%%%%%
\section{Citations}

\begin{thebibliography}{9}

\bibitem{labpacket}
  Karen Schnurbusch,
  \textit{Physics 4B Lab Book},
  Mt. San Antonio College,
  2023,
  pp. 65-70.

\bibitem{equationsheet}
  Karen Schnurbusch,
  \textit{Physics 4B Equations},
  Mt. San Antonio College,
  2023,
  pp. 4, 10.

\end{thebibliography}

\end{document}
%%%%%%%%%%%%%%%%%%%% Document Ends %%%%%%%%%%%%%%%%%%%%
