\documentclass[12pt]{iopart} % Document class declaration

% package "imports"
\usepackage{graphicx}
\usepackage{IEEEtrantools}

% Custom macros
\gdef\mcm{r@{.}l@{ ± }r@{.}l} % Multi Column Measurement; Used for decimal aligning & ± aligning
\gdef\mch#1{\multicolumn{4}{l}{#1}} % Multi Column Header; Used for decimal aligning & ± aligning
\gdef\mcmnd{r@{ ± }l} % Multi Column Measurement No Decimal; Used for ± aligning when the values don't need a decimal point
\gdef\mchnd#1{\multicolumn{2}{l}{#1}} % Multi Column Header No Decimal; Used for  ± aligning when the values don't need a decimal point
\gdef\sci#1#2{#1 \times 10^{#2}}
\gdef\units#1{~\mathrm{#1}}

%%%%%%%%%%%%%%%%%%%% Document Starts %%%%%%%%%%%%%%%%%%%%
\begin{document}

%%%%%%%%%%%%%%%%%%%% Title Page %%%%%%%%%%%%%%%%%%%%
\title{Resistance Lab}
\author{Vincent Edwards, Ali Mortada, Andrew Bringas}
\vspace{10pt}
\begin{indented}
  \item[]Mt.~San Antonio College, Physics 4B, CRN 42240
  \item[]April 26, 2023
\end{indented}
\newpage

%%%%%%%%%%%%%%%%%%%% Purpose %%%%%%%%%%%%%%%%%%%%
\section{Purpose}

For part 1, the goal was to examine how material, cross-sectional area, and length impact resistance.
For each coil of wire, resistance was determined in three ways: based on measurements of voltage and current; based on the resistance readings from the multimeter; and based on the dimensions of the wires and the resistivity of the materials.
In addition, the results of the three methods of determining resistance were compared.

%%%%%%%%%%%%%%%%%%%% Results %%%%%%%%%%%%%%%%%%%%
\section{Results}

Table \ref{tab:resistance_coils} contains the properties of the resistance coils used in part 1.
$\rho$ is the resistivity of the material.
$L$ is the length of the coil of wire.
$D$ is the diameter of the wire.
The values for $\rho$, $L$, and $D$ were assumed to be exact, without uncertainty.
In addition, note that Nickel-Silver is a copper alloy consisting of copper, nickel, and zinc or manganese.
The exact composition of the alloy used in the experiment is unknown.
Thus, the resistivity was approximated to be $\sci{44}{-8} \units{\Omega~m}$, the value for Manganin (Cu 84\%, Mn 12\%, Ni 4\%), an alloy of similar composition.

\begin{table}[htbp]
\caption{\label{tab:resistance_coils}
Resistance Coils
}
\begin{indented}\lineup\item[]\begin{tabular}{@{}llrrr}
\br
  Coil & Material      & $\rho~(\mathrm{\Omega~m})$ & $L$ (cm) & $D$ (cm) \\
\mr
  1    & Nickel-Silver & $\sci{44}{-8}$             &       40 & 0.0254 \\
  2    & Nickel-Silver & $\sci{44}{-8}$             &       80 & 0.0254 \\
  3    & Nickel-Silver & $\sci{44}{-8}$             &      120 & 0.0254 \\
  4    & Nickel-Silver & $\sci{44}{-8}$             &      160 & 0.0254 \\
  5    & Nickel-Silver & $\sci{44}{-8}$             &      200 & 0.0254 \\
  6    & Nickel-Silver & $\sci{44}{-8}$             &      200 & 0.0320 \\
  7    & Copper        & $\sci{1.72}{-8}$           &     2000 & 0.0254 \\
\br
\end{tabular}\end{indented}\end{table}

Table \ref{tab:measurements_part_1} contains the measurements made during part 1.
$\Delta V$ is the voltage across the resistance coil.
$I$ is the current through the resistance coil.
($R_\mathrm{m} + r$) is the resistance, measured using the multimeter, of the resistance coil and the wires connecting it to the multimeter.
$r$ is the resistance, measured using the multimeter to be $0.1 \pm 0.1 \units{\Omega}$, of just the wires used for connecting the multimeter to the resistance coil.

\begin{table}[htbp]
\caption{\label{tab:measurements_part_1}
Part 1 Measurements \\
Note: $r$, the resistance of the two wires connected to the multimeter, was measured to be $0.1 \pm 0.1 \units{\Omega}$.
}
\begin{indented}\lineup\item[]\begin{tabular}{@{}l\mcm\mcm\mcm}
\br
  Coil & \mch{$\Delta V$ (V)} & \mch{$I$ (mA)}   & \mch{$R_\mathrm{m} + r$ (Ω)} \\
\mr
  1    & 0&214 & 0&001        & 55&07 & 0&05     & 4&0 & 0&1        \\
  2    & 0&400 & 0&001        & 49&72 & 0&01     & 8&3 & 0&1        \\
  3    & 0&536 & 0&001        & 44&53 & 0&02     & 12&0 & 0&1       \\
  4    & 0&662 & 0&001        & 41&70 & 0&01     & 15&9 & 0&1       \\
  5    & 0&772 & 0&001        & 38&19 & 0&01     & 20&3 & 0&1       \\
  6    & 0&538 & 0&001        & 45&44 & 0&01     & 11&7 & 0&1       \\
  7    & 0&357 & 0&001        & 51&31 & 0&01     & 7&3 & 0&1        \\
\br
\end{tabular}\end{indented}\end{table}

Table \ref{tab:resistances_part_1} contains the two experimental values for resistance in the coil ($R_{V/I}$ and $R_{\mathrm{m}}$) and the theoretical value ($R_{\mathrm{th}}$).
$R_{V/I}$ is the experimental resistance determined using measured values for $\Delta V$ and $I$.
$R_{\mathrm{m}}$ is the experimental resistance determined using the readings from the multimeter.
$R_{\mathrm{th}}$ is the theoretical resistance determined using the dimensions of the wires ($L$ and $D$) and the resistivity of the material ($\rho$).

\begin{table}[htbp]
\caption{\label{tab:resistances_part_1}
Part 1 Experimental and Theoretical Resistances
}
  \begin{indented}\lineup\item[]\begin{tabular}{@{}l\mcm\mcm r@{.}l}
\br
  Coil & \mch{$R_{V/I}$ (Ω)} & \mch{$R_{\mathrm{m}}$ (Ω)} & \multicolumn{2}{l}{$R_{\mathrm{th}}$ (Ω)} \\
\mr
  1    & 3&89 & 0&02         & 3&9 & 0&1                  & 3&47      \\
  2    & 8&05 & 0&02         & 8&2 & 0&1                  & 6&95      \\
  3    & 12&04 & 0&02        & 11&9 & 0&1                 & 10&42     \\
  4    & 15&88 & 0&02        & 15&8 & 0&1                 & 13&89     \\
  5    & 20&21 & 0&03        & 20&2 & 0&1                 & 17&37     \\
  6    & 11&84 & 0&02        & 11&6 & 0&1                 & 10&94     \\
  7    & 6&96 & 0&02         & 7&2 & 0&1                  & 6&79      \\
\br
\end{tabular}\end{indented}\end{table}

Figure \ref{fig:resistance_vs_length} plots resistance ($R_{V/I}$, $R_\mathrm{m}$, and $R_\mathrm{th}$) versus length ($L$) for the first five resistance coils used in part 1.
Those coils were chosen for the graph as they all had the same material and diameter.
Since the graphs seemed linear (and that's what theory predicts) a linear fit was applied.

\begin{figure}[htbp]
  \begin{indented}
  \item[]\includegraphics[width=0.83\textwidth]{resistance-vs-length-graph.png}
  \end{indented}
  \caption{\label{fig:resistance_vs_length}
  Resistance vs Length (Constant $\rho$ and $A$)
  }
\end{figure}

%%%%%%%%%%%%%%%%%%%% Uncertainty %%%%%%%%%%%%%%%%%%%%
\section{Uncertainty}

All the voltage, current, and resistance measurements in the experiment were made using a multimeter.
If there were fluctuations in a particular reading with the multimeter (voltage, current, or resistance), then the uncertainty was taken to be the magnitude of the fluctuations, as that gives an upper bound for how much the value could have varied.
If the reading with the multimeter (voltage, current, or resistance) was stable and did not fluctuate, then the uncertainty was taken to be the smallest increment of measure, since that was the limiting precision of the equipment.

In part 1, $R_{V/I}$ was calculated using equation \ref{eq:r_from_vi}.
\begin{IEEEeqnarray}{rCl}
  R_{V/I} & = & \frac{\Delta V}{I} \label{eq:r_from_vi}
\end{IEEEeqnarray}
The uncertainty of $R_{V/I}$ is given by equation \ref{eq:r_from_vi_uncertainty}.
\begin{IEEEeqnarray}{rCl}
  \Delta R_{V/I} & = & \left[
    \left(\frac{\partial R_{V/I}}{\partial \Delta V} \Delta \Delta V \right)^2
    + \left(\frac{\partial R_{V/I}}{\partial I} \Delta I \right)^2
  \right]^\frac{1}{2} \label{eq:r_from_vi_uncertainty} \\
  & = & \frac{\Delta V}{I} \left[
    \left( \frac{\Delta \Delta V}{\Delta V} \right)^2
    + \left( \frac{\Delta I}{I} \right)^2
  \right]^\frac{1}{2} \nonumber
\end{IEEEeqnarray}

In part 1, $R_\mathrm{m}$ was calculated using equation \ref{eq:r_from_m}.
Note that the combined value ($R_\mathrm{m} + r$) was measured, as was the value for $r$.
But $R_\mathrm{m}$ was not measured, hence why it is being calculated as a difference of two values.
\begin{IEEEeqnarray}{rCl}
  R_\mathrm{m} & = & (R_\mathrm{m} + r) - r \label{eq:r_from_m}
\end{IEEEeqnarray}
The uncertainty of $R_\mathrm{m}$ is given by equation \ref{eq:r_from_m_uncertainty}.
\begin{IEEEeqnarray}{rCl}
  \Delta R_\mathrm{m} & = & \left[
    \left(\frac{\partial R_\mathrm{m}}{\partial (R_\mathrm{m} + r)} \Delta (R_\mathrm{m} + r) \right)^2
    + \left(\frac{\partial R_\mathrm{m}}{\partial r} \Delta r \right)^2
  \right]^\frac{1}{2} \label{eq:r_from_m_uncertainty} \\
  & = & \left[
    \left( \Delta (R_\mathrm{m} + r) \right)^2
    + \left( \Delta r \right)^2
  \right]^\frac{1}{2} \nonumber
\end{IEEEeqnarray}

In part 1, $R_\mathrm{th}$ was calculated using equation \ref{eq:r_th}.
\begin{IEEEeqnarray}{rCl}
  R_\mathrm{th} & = & \frac{\rho L}{A} \label{eq:r_th} \\
  & = & \frac{\rho L}{\pi (D/2)^2} \nonumber \\
  & = & \frac{4 \rho L}{\pi D^2} \nonumber
\end{IEEEeqnarray}
Note that the values for $\rho$, $L$, and $D$ were assumed to be exact, without uncertainty.
Thus, $R_\mathrm{th}$ would have no uncertainty as well.

%%%%%%%%%%%%%%%%%%%% Conclusion %%%%%%%%%%%%%%%%%%%%
\section{Conclusion}

In part 1, the resistances of 7 different resistance coils were determined using three different techniques.
Descriptions of the coils are in table \ref{tab:resistance_coils}, and the calculated resistance values are in table \ref{tab:resistances_part_1}.
The experimental resistances $R_{V/I}$ and $R_\mathrm{m}$ had similar values for all the coils.
Coils 1, 4, and 5 had resistance values that overlapped within uncertainty, while the rest of the coils (2, 3, 6, and 7) had values that were still fairly close and almost within uncertainty.
While the experimental resistances were similar, their values were all higher than the corresponding theoretical resistances $R_\mathrm{th}$.
One possible explanation is that when measuring voltage and resistance, the multimeter was connected to the resistance coils through a small bit of copper wire.
This copper wire section was not included when measuring $r$, but it was included when measuring $(R_\mathrm{m} + r)$.
As a result, this would cause $R_\mathrm{m}$ to be higher than expected.
This extra resistance would also have increased the measured value for $\Delta V$, thereby causing $R_{V/I}$ to be higher than expected.
Another possibility is that the temperature of the room did not match the reference temperature of 20 ℃.
The table the resistivity values were taken from used a reference temperature of 20 ℃.
If the room were warmer than 20 ℃, then the actual resistivity values would have been slightly higher than the $\rho$ values used for calculations.
This would cause $R_\mathrm{th}$ to be lower than expected.

%%%%%%%%%%%%%%%%%%%% Citations %%%%%%%%%%%%%%%%%%%%
\section{Citations}

\begin{thebibliography}{9}

\bibitem{labpacket}
  Karen Schnurbusch,
  \textit{Physics 4B Lab Book},
  Mt. San Antonio College,
  2023,
  pp. 65-70.

\bibitem{equationsheet}
  Karen Schnurbusch,
  \textit{Physics 4B Equations},
  Mt. San Antonio College,
  2023,
  pp. 4, 10.

\end{thebibliography}

\end{document}
%%%%%%%%%%%%%%%%%%%% Document Ends %%%%%%%%%%%%%%%%%%%%
