\documentclass[12pt]{iopart} % Document class declaration

% package "imports"
\usepackage{graphicx}
\usepackage{IEEEtrantools}

% Custom macros
\gdef\mcm{r@{.}l@{ ± }r@{.}l} % Multi Column Measurement; Used for decimal aligning & ± aligning
\gdef\mch#1{\multicolumn{4}{l}{#1}} % Multi Column Header; Used for decimal aligning & ± aligning
\gdef\mcmnd{r@{ ± }l} % Multi Column Measurement No Decimal; Used for ± aligning when the values don't need a decimal point
\gdef\mchnd#1{\multicolumn{2}{l}{#1}} % Multi Column Header No Decimal; Used for  ± aligning when the values don't need a decimal point
\gdef\sci#1#2{#1 \times 10^{#2}}
\gdef\units#1{~\mathrm{#1}}

\gdef\emf{\mathcal{E}}

%%%%%%%%%%%%%%%%%%%% Document Starts %%%%%%%%%%%%%%%%%%%%
\begin{document}

%%%%%%%%%%%%%%%%%%%% Title Page %%%%%%%%%%%%%%%%%%%%
\title{Circuits Lab}
\author{Vincent Edwards}
\vspace{10pt}
\begin{indented}
  \item[]Mt.~San Antonio College, Physics 4B, CRN 42240
  \item[]May 8, 2023
  \item[]
  \item[]$R_2 = 40 \units{\Omega}$
\end{indented}
\newpage

%%%%%%%%%%%%%%%%%%%% Purpose %%%%%%%%%%%%%%%%%%%%
\section{Purpose}

The goal of the exercise was to use Kirchhoff's rules to analyze 7 different circuits and calculate current, voltage, and power for each circuit element.
For the first 4 circuits, the voltage and current calculations were compared to results obtained using an online circuit simulator.

%%%%%%%%%%%%%%%%%%%% Results %%%%%%%%%%%%%%%%%%%%
\section{Results}

The following tables contain the theoretical voltage, current, and power for each circuit element for each of the 7 circuits.
In addition, there is an annotated circuit diagram for each circuit.

\begin{figure}[htbp]
  \begin{indented}
  \item[]\includegraphics[width=0.83\textwidth]{circuit-1.png}
  \end{indented}
  \caption{\label{fig:circuit_1}
  Circuit 1
  }
\end{figure}

\begin{table}[htbp]
\caption{\label{tab:circuit_1}
Circuit 1 $V$, $I$, and $P$
}
\begin{indented}\lineup\item[]\begin{tabular}{@{}lr@{.}lr@{.}lr@{.}l}
\br
  & \multicolumn{2}{l}{$V \units{(V)}$} & \multicolumn{2}{l}{$I \units{(A)}$} & \multicolumn{2}{l}{$P \units{(W)}$} \\
\mr
  $\emf_1$ & 12&0 & 0&0440 & 0&527 \\
  $R_1$    & 4&84 & 0&0440 & 0&213 \\
  $R_2$    & 1&76 & 0&0440 & 0&0773 \\
  $R_3$    & 1&98 & 0&0440 & 0&0869 \\
  $R_4$    & 3&43 & 0&0440 & 0&151 \\
\br
\end{tabular}\end{indented}\end{table}

\begin{figure}[htbp]
  \begin{indented}
  \item[]\includegraphics[width=0.83\textwidth]{circuit-2.png}
  \end{indented}
  \caption{\label{fig:circuit_2}
  Circuit 2
  }
\end{figure}

\begin{table}[htbp]
\caption{\label{tab:circuit_2}
Circuit 2 $V$, $I$, and $P$
}
\begin{indented}\lineup\item[]\begin{tabular}{@{}lr@{.}lr@{.}lr@{.}l}
\br
  & \multicolumn{2}{l}{$V \units{(V)}$} & \multicolumn{2}{l}{$I \units{(A)}$} & \multicolumn{2}{l}{$P \units{(W)}$} \\
\mr
  $\emf_1$ & 9&00 & 0&403 & 3&63 \\
  $R_1$    & 9&00 & 0&106 & 0&953 \\
  $R_2$    & 9&00 & 0&225 & 2&02 \\
  $R_3$    & 9&00 & 0&0720 & 0&648 \\
\br
\end{tabular}\end{indented}\end{table}

\begin{figure}[htbp]
  \begin{indented}
  \item[]\includegraphics[width=0.83\textwidth]{circuit-3.png}
  \end{indented}
  \caption{\label{fig:circuit_3}
  Circuit 3
  }
\end{figure}

\begin{table}[htbp]
\caption{\label{tab:circuit_3}
Circuit 3 $V$, $I$, and $P$
}
\begin{indented}\lineup\item[]\begin{tabular}{@{}lr@{.}lr@{.}lr@{.}l}
\br
  & \multicolumn{2}{l}{$V \units{(V)}$} & \multicolumn{2}{l}{$I \units{(A)}$} & \multicolumn{2}{l}{$P \units{(W)}$} \\
\mr
  $\emf_1$ & 11&0 & 0&212 & 2&34 \\
  $R_1$    & 11&0 & 0&0815 & 0&896 \\
  $R_2$    & 5&24 & 0&131 & 0&686 \\
  $R_3$    & 5&76 & 0&0703 & 0&405 \\
  $R_4$    & 5&76 & 0&0607 & 0&350 \\
\br
\end{tabular}\end{indented}\end{table}

\begin{figure}[htbp]
  \begin{indented}
  \item[]\includegraphics[width=0.83\textwidth]{circuit-4.png}
  \end{indented}
  \caption{\label{fig:circuit_4}
  Circuit 4
  }
\end{figure}

\begin{table}[htbp]
\caption{\label{tab:circuit_4}
Circuit 4 $V$, $I$, and $P$
}
\begin{indented}\lineup\item[]\begin{tabular}{@{}lr@{.}lr@{.}lr@{.}l}
\br
  & \multicolumn{2}{l}{$V \units{(V)}$} & \multicolumn{2}{l}{$I \units{(A)}$} & \multicolumn{2}{l}{$P \units{(W)}$} \\
\mr
  $\emf_1$ & 10&0 & 0&0613 & 0&613 \\
  $R_1$    & 2&80 & 0&0280 & 0&0783 \\
  $R_2$    & 1&33 & 0&0333 & 0&0443 \\
  $R_3$    & 2&73 & 0&0333 & 0&0907 \\
  $R_4$    & 1&26 & 0&0280 & 0&0353 \\
  $R_5$    & 5&94 & 0&0613 & 0&364 \\
\br
\end{tabular}\end{indented}\end{table}

\begin{figure}[htbp]
  \begin{indented}
  \item[]\includegraphics[width=0.83\textwidth]{circuit-5.png}
  \end{indented}
  \caption{\label{fig:circuit_5}
  Circuit 5
  }
\end{figure}

\begin{table}[htbp]
\caption{\label{tab:circuit_5}
Circuit 5 $V$, $I$, and $P$
}
\begin{indented}\lineup\item[]\begin{tabular}{@{}lr@{.}lr@{.}lr@{.}l}
\br
  & \multicolumn{2}{l}{$V \units{(V)}$} & \multicolumn{2}{l}{$I \units{(A)}$} & \multicolumn{2}{l}{$P \units{(W)}$} \\
\mr
  $\emf_1$ & 14&0 & 0&0750 & 1&05 \\
  $\emf_2$ & 5&00 & 0&0750 & 0&375 \\
  $R_1$    & 2&62 & 0&0750 & 0&197 \\
  $R_2$    & 3&00 & 0&0750 & 0&225 \\
  $R_3$    & 3&38 & 0&0750 & 0&253 \\
\br
\end{tabular}\end{indented}\end{table}

\begin{figure}[htbp]
  \begin{indented}
  \item[]\includegraphics[width=0.83\textwidth]{circuit-6.png}
  \end{indented}
  \caption{\label{fig:circuit_6}
  Circuit 6
  }
\end{figure}

\begin{table}[htbp]
\caption{\label{tab:circuit_6}
Circuit 6 $V$, $I$, and $P$
}
\begin{indented}\lineup\item[]\begin{tabular}{@{}lr@{.}lr@{.}lr@{.}l}
\br
  & \multicolumn{2}{l}{$V \units{(V)}$} & \multicolumn{2}{l}{$I \units{(A)}$} & \multicolumn{2}{l}{$P \units{(W)}$} \\
\mr
  $\emf_1$ & 40&0 & 2&11 & 84&6 \\
  $\emf_2$ & 10&0 & 0&343 & 3&43 \\
  $R_1$    & 24&3 & 0&971 & 23&6 \\
  $R_2$    & 13&7 & 0&343 & 4&70 \\
  $R_3$    & 40&0 & 1&14 & 45&7 \\
  $R_4$    & 15&7 & 0&629 & 9&88 \\
  $R_5$    & 12&0 & 0&343 & 4&11 \\
\br
\end{tabular}\end{indented}\end{table}

\begin{figure}[htbp]
  \begin{indented}
  \item[]\includegraphics[width=0.83\textwidth]{circuit-7.png}
  \end{indented}
  \caption{\label{fig:circuit_7}
  Circuit 7
  }
\end{figure}

\begin{table}[htbp]
\caption{\label{tab:circuit_7}
Circuit 7 $V$, $I$, and $P$
}
\begin{indented}\lineup\item[]\begin{tabular}{@{}lr@{.}lr@{.}lr@{.}l}
\br
  & \multicolumn{2}{l}{$V \units{(V)}$} & \multicolumn{2}{l}{$I \units{(A)}$} & \multicolumn{2}{l}{$P \units{(W)}$} \\
\mr
  $\emf_1$ & 12&0 & 0&200 & 2&40 \\
  $R_1$    & 2&00 & 0&200 & 0&399 \\
  $R_2$    & 2&58 & 0&0646 & 0&167 \\
  $R_3$    & 4&05 & 0&135 & 0&547 \\
  $R_4$    & 2&58 & 0&0646 & 0&167 \\
  $R_5$    & 5&95 & 0&119 & 0&709 \\
  $R_6$    & 4&83 & 0&0806 & 0&390 \\
  $R_7$    & 1&12 & 0&0160 & 0&0179 \\
\br
\end{tabular}\end{indented}\end{table}

%%%%%%%%%%%%%%%%%%%% Calculations %%%%%%%%%%%%%%%%%%%%
\section{Calculations}

%%%%%%%%%%%%%%%%%%%% Conclusion %%%%%%%%%%%%%%%%%%%%
\section{Conclusion}

%%%%%%%%%%%%%%%%%%%% Citations %%%%%%%%%%%%%%%%%%%%
\section{Citations}

\begin{thebibliography}{9}

\bibitem{labpacket}
  Karen Schnurbusch,
  \textit{Physics 4B Lab Book},
  Mt. San Antonio College,
  2023,
  pp. 71-74.

\bibitem{equationsheet}
  Karen Schnurbusch,
  \textit{Physics 4B Equations},
  Mt. San Antonio College,
  2023,
  pp. 4, 5.

\end{thebibliography}

\end{document}
%%%%%%%%%%%%%%%%%%%% Document Ends %%%%%%%%%%%%%%%%%%%%
